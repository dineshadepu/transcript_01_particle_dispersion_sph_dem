\documentclass[preprint,12pt]{elsarticle}
% \documentclass[preprint,12pt,draft]{elsarticle}
% \documentclass[draft,12pt]{elsarticle}

\usepackage{hyperref}
\usepackage{graphicx}
\usepackage{subcaption}
\usepackage{amssymb}
\usepackage{amsmath}
\usepackage{multirow}
\usepackage{relsize}
\usepackage[utf8]{inputenc}
\usepackage[capitalise]{cleveref}
\usepackage{algorithm}
\usepackage[noend]{algpseudocode}
\usepackage[section]{placeins}
\usepackage{booktabs}
\usepackage{url}

% For the TODOs
\usepackage{xcolor}
\usepackage{xargs}
\usepackage[colorinlistoftodos,textsize=footnotesize]{todonotes}
\newcommand{\todoin}{\todo[inline]}
% from here: https://tex.stackexchange.com/questions/9796/how-to-add-todo-notes
\newcommandx{\unsure}[2][1=]{\todo[linecolor=red,backgroundcolor=red!25,bordercolor=red,#1]{#2}}
\newcommandx{\change}[2][1=]{\todo[linecolor=blue,backgroundcolor=blue!25,bordercolor=blue,#1]{#2}}
\newcommandx{\info}[2][1=]{\todo[linecolor=OliveGreen,backgroundcolor=OliveGreen!25,bordercolor=OliveGreen,#1]{#2}}

%Boldtype for greek symbols
\newcommand{\teng}[1]{\ensuremath{\boldsymbol{#1}}}
\newcommand{\ten}[1]{\ensuremath{\mathbf{#1}}}

\usepackage{lineno}
% \linenumbers

\journal{}

\begin{document}

\begin{frontmatter}

  \title{{DEM}-{SPH} study of particle dispersion in fluid}
  \author[xxx]{Dinesh Adepu\corref{cor1}}
  \ead{d.adepu@surrey.ac.uk}
  \author[xxx]{Chuan Yu Wu}
  % \ead{XXX}
\address[xxx]{School of Chemistry and Chemical Engineering, University of Surrey, Guildford, GU2 7XH, UK}

\cortext[cor1]{Corresponding author}


\begin{abstract}
  Mixing powdered substances in stirring tanks is a common occurrence across
  various industries. In our current study, we developed a solver combining
  Smoothed Particle Hydrodynamics (SPH) and Discrete Element Method (DEM) to
  investigate the mixing behavior of particles in a tank under the influence
  of a stirrer operating at different speeds.  The SPH method governs the
  fluid phase, while the dynamics and interactions of particles are captured
  by DEM. We achieve a fully resolved coupling between solid particles and
  fluid particles by discretizing solid particles into dummy SPH
  particles. Validation was performed, including verifying the fluid solver's
  accuracy through a Poiseuille problem and validating the DEM solver by
  benchmarking interactions between two particles and particle-wall impacts.
  The coupled model is validated by simulating single-particle entry in a
  steady tank in 2D and a cube settling in a tank in 3D. Using the validated
  model, we conduct two simulations: one with spherical particles of uniform
  radius and another with particles having two different radius.  Our results
  reveal that at lower stirrer speeds, particles tend to aggregate
  initially and remain in the center. However, with the increase in
  stirrer speed, particles tend to accumulate near the tank corners due to the
  circulation induced in the fluid due to the stirrer motion.
\end{abstract}

\begin{keyword}
%% keywords here, in the form: keyword \sep keyword
{particle dispersion}, {particle mixing}, {SPH-DEM}, {stirrer}

%% MSC codes here, in the form: \MSC code \sep code
%% or \MSC[2008] code \sep code (2000 is the default)

\end{keyword}

\end{frontmatter}

% \linenumbers


\FloatBarrier%
\section{Introduction}

Particle dispersion is a fundamental physical phenomenon encountered in
various industries where powder and fluid interactions are pivotal.  Control
of water flooding in petroleum industry \cite{wang2023developments}, the
transport of bodies in internal systems \cite{Dai2021}, debris flow
\cite{Qingyun2022}, the food processing industry \cite{Karunasena2014}, and
ice-sea modeling \cite{Mintu2018} are a few examples to mention.  In many
processes, the mixing of powder with fluid using a stirrer are commonplace
\cite{li2022study}.  \Cref{intro:schematic} illustrates a sample particle
dispersion problem.  While experimental investigation of such processes is
often impractical due to their highly nonlinear nature, numerical methods
offer a viable alternative.  While these systems are studied with numerical
methods, they are generally treated as a two-way coupling problem. In
numerical method approach, one can employ either a mesh-based or meshless
technique to study these phenomena.

\begin{figure}[!htpb]
  \centering
  \includegraphics[width=0.4\textwidth]{images/hs_tank_fluid_with_particles}
  \caption{An illustrative figure showcasing the dispersion of particles within a free surface tank.}
  \label{intro:schematic}
\end{figure}



The combination of Discrete Element Method (DEM) for solid particle
interactions and Computational Fluid Dynamics (CFD) for fluid modeling is a
classic approach in handling particulate flow problem. In mesh-based modeling,
lattice Boltzmann method (LBM) \cite{xiong2014lbm} and finite volume method
(FVM) \cite{kloss2012models} are utilized to handle the fluid
dynamics. Similarly, in meshless schemes, Smoothed Particle Hydrodynamics
(SPH) \cite{peng2021fully}, Moving Particle Semi-implicit (MPS), and Particle
Finite Element Method (PFEM) \cite{li2019modeling, franci2020pfem}, are
adopted to handle the fluid dynamics.


While handling the interaction between the fluid and solid particles, the
interaction can be categorized into a resolved and unresolved coupling.  In
unresolved coupling, simplified models are utilized to calculate the forces
exerted on solid particles within the fluid flow. In contrast, in the fully
resolved coupling approach, the fluid forces are directly computed on the
solid particles without the need for empirical drag force models.  Both
strategies are applicable in both meshless \cite{cleary2015prediction,
  trujillo2020smooth} and mesh-based \cite{van2008numerical, ma2022review}
techniques.  In addition to study mono-sized particle behaviour in a fluid
\cite{trujillo2020smooth}, research, such as \cite{brosh2014accelerating} and
\cite{peng2021fully}, also explore wide particle size distributions with
CFD-DEM and SPH-DEM respectively.  Regarding non-Newtonian fluid modelling in
mesh-based schemes, works like \cite{li2018dam} are notable and in meshless
schemes, \cite{peng2021fully} is notable. \citet{zhu2008discrete} and
\citet{ma2022review} delve into investigations on spherical and non-spherical
particles respectively within CFD-DEM frameworks.  Mesh based CFD-DEM, though
fruitful, encounters challenges in handling problems with free surfaces and
becomes computationally demanding with a large number of solid particles,
especially in fully resolved coupling. In contrast, SPH offers advantages in
such scenarios due to its inherent capability in handling free surfaces and
adeptness in modeling large mesh deformation problems, facilitating the
incorporation of novel physics.


The SPH method is a meshless numerical method originally proposed by
\citet{gingold1977smoothed} and \citet{lucy1977numerical} to model
astrophysics problems. It has been extensively applied to simulate problems
involving fluids, structural dynamics, fluid-structure interaction, granular
physics, non-Newtonian fluid flows \cite{peng2021fully} among other areas
\cite{monaghan2012smoothed}.  Various kind of particulate flows, including,
simple spherical particles mixing using unresolved coupling
\cite{cleary2015prediction, markauskas2019coupled} and with debris flow
\cite{canelas2016sph} with fully resolved coupling approach, were analysed
using SPH-DEM.  Handling particles of arbitrary shape in fluid flow is
addressed in various works, including those by \citet{peng2021fully,
  canelas2016sph}, and \citet{amicarelli2015smoothed}.  Several variants of
SPH, such as weakly compressible SPH (WCSPH) \cite{peng2021fully,
  cleary2015prediction}, and ISPH \cite{asai2021fluid}, are utilized for
modeling fluid flow with particulate flows. The coupling between fluid and
rigid bodies in the fully resolved category employs techniques such as the
fixed ghost particle technique \cite{canelas2016sph, asai2021fluid}, single
layers of dummy SPH particles \cite{peng2021fully}, and simple repulsive
forces \cite{monaghan2009sph}. Coupling between SPH fluid particles and solid
structures discretized as vertex edges is also proposed by
\citet{park2023new}. Particle settling with variable sizes is studied by
\citet{zou2022study}.  Elastic behavior of spherical particles and dispersion
studies are conducted by \citet{ng2021numerical}. Interaction between rigid
bodies of arbitrary sizes is addressed using various techniques. For example,
\citet{asai2021fluid} utilize an energy-tracking impulse method, while
\citet{peng2021fully} apply a surface mesh-based Discrete Element Method
(SMR-DEM) to handle collisions.


Mixing of particles in a tank with stirrer is modelled with an unresolved
SPH-DEM model by \citet{zhu2021virtual}. While the interaction between two
solid particles is handled using a linear model. A similar study is carried
out to study the mixing of particles in a rotating drum by \citet{he2018gpu}
using unresolved SPH-DEM model. Here, the authors use non-linear contact force
model, which captures the force model accurately.  While using a fully
resolved SPH-DEM coupling solver, particle dispersion has been studied in
\cite{ng2021numerical,canelas2016sph,canelas2017resolved,peng2021fully}, where
the interaction between the solid particle and the fluid is handled using
ghost particles placed on the solid particle. In works of unresolved coupling,
the interaction among the solid particles is handled using the discretized
ghost points with a modified DEM model. However, fully resolved coupled works
have only been applied to works with arbitrary shaped bodies, and to mainly
ocean engineering problems \cite{canelas2016sph,canelas2017resolved} but not
for particle mixing in tank.  In the current work, we study the circular
particle mixing in stirring tanks.  We employ a fully resolved coupled SPH-DEM
model, however, unlike previous fully resolved approaches
\cite{ng2021numerical,canelas2016sph,canelas2017resolved,peng2021fully}, we
only utilize the discretized ghost particles placed on the circular solid
particle for interaction with the fluid, but not for solid solid
interaction. The solid-solid interaction is carried out using the traditional
DEM methodology \cite{cundall_discrete_1979}. We validate the fluid solver
independently using Poiseuille flow and the DEM solver separately using
fundamental benchmarks such as particle-particle and particle-wall
impacts. Subsequently, the fully resolved coupled SPH-DEM solver is validated.

Upon confirming the fidelity of the solver, we proceed with analyzing the
mixing behavior of various spherical particles influenced by a stirrer in a
tank. We investigate scenarios involving particles of different densities,
stirrer velocities, and variable sizes as part of our case study. The
development work is conducted using PySPH\cite{ramachandran2021pysph}, an
open-source code available at \url{https://github.com/pypr/pysph}.  We modify
the current codebase to integrate implementations of the DEM and the SPH
coupled with DEM model for our ongoing research. To ensure reproducibility, we
utilize the automan package \cite{ramachandran2018automan} to automate all
results generated in the current manuscript, aiming for a reproducible
research approach.




% \begin{itemize}
% \item
% \item \cite{zhu2021virtual} uses unresolved coupling, with contact between the
%   particles being a linear model. Mixing in a tank with stirrer.
% \item \cite{he2018gpu} uses unresolved coupling, with contact between the
%   particles being a Hertz model, considering the real material properties of
%   the interacting particles. Rotating drum mixing.
% \item tong2018scale uses unresolved coupling, with contact between the
%   particles being a linear model. Studies mixing in tank.
% \item Circular particle dispersion study SPH-DEM-VCPM studies
%   \citet{ng2021numerical}, Fully resolved coupling, with particles interacting
%   among themselves using a linear force model, the interaction between two
%   solid particles is handled using the dummy points of VCPM.
% \item Circular particle dispersion study SPH-DEM-VCPM studies
%   \citet{ng2021numerical}, Fully resolved coupling, with particles interacting
%   among themselves using a linear force model, the interaction between two
%   solid particles is handled using the dummy points of VCPM.
% \item canelas2016sph, canelas2017resolved, peng2021fully model transport of
%   rigid body of circular and arbitrary shape in a fluid flow, mainly applied to
%   areas of ocean engineering and not anything related to spherical particle
%   mixing.  \citet{ng2021numerical} assumes the particles to be elastic and
%   models it with VCPM method, and those discretized VCPM notes also acts as
%   SPH dummy particles and rigid fluid coupling is carried out using that.
% \item Though fully resolved coupling works have been existing in the
%   literature the interaction among the solid particles are handled through the
%   dummy particles being discretized. Using the ghost particles, the contact
%   force between the solid particles has been handled, where both linear and
%   nonlinear contact force models are studied. While the contact force
%   coefficients are calibrated for the distributed contact (DCDEM).
% \item In the current work, we develop a coupled SPH-DEM solver, to model the
%   particle dispersion within a large-scale under stirrer mixing phenomenon.
% \item We adopt a fully coupled approach to model the interaction between the
%   solid particles and the fluid unlike the unresolved studies.
% \item The interaction between the solid particles is handled using the DEM
%   model, however, we will use the center of the solid bodies to handle the
%   contact, and not the dummy particles as used in \cite{}canelas2016sph,
%   \cite{canelas2017resolved}, \cite{peng2021fully}. This provides us with more accurate
%   contact force modeling, with a nonlinear force model.
% \end{itemize}



% In the current work we consider the particle dispersion within a large-scale
% under stirrer mixing phenomenon. We employ Smoothed Particle Hydrodynamics
% (SPH) to model fluid dynamics and the Discrete Element Method (DEM) to handle
% the dynamics and contact interactions among spherical particles. The
% interaction between solids and fluids is modelled using a fixed ghost particle
% approach, ensuring a fully resolved simulation. We validate the fluid solver
% independently using Poiseuille flow and the DEM solver separately using
% fundamental benchmarks such as particle-particle and particle-wall
% impacts. Subsequently, the fully resolved coupled SPH-DEM solver is validated
% using three benchmarks: a single particle entering, two particles settling,
% and a forced wedge entry into a steady hydrostatic tank. Upon confirming the
% fidelity of the solver, we proceed with analyzing the mixing behavior of
% various spherical particles influenced by a stirrer in a tank. We investigate
% scenarios involving particles of different densities, stirrer velocities, and
% variable sizes as part of our case study. The development work is conducted
% using PySPH\cite{ramachandran2021pysph}, an open-source code available at
% \url{https://github.com/pypr/pysph}.  We modify the current codebase to
% integrate implementations of the Discrete Element Method (DEM) and the
% Smoothed Particle Hydrodynamics (SPH) coupled with DEM model for our ongoing
% research. To ensure reproducibility, we utilize the automan package
% \cite{ramachandran2018automan} to automate all results generated in the
% current manuscript, aiming for a reproducible research approach.



The structure of this paper is as follows: In \cref{sec:fluid-modeling}, we
explain the numerical method used to model fluid dynamics. \Cref{sec:rbd}
outlines the equations governing the dynamics of rigid bodies and the contact
force model employed to resolve collisions among them. The coupling between
rigid spherical particles and the fluid is detailed in
\cref{sec:rfc}. \cref{sec:integration} discusses the time integration scheme
employed in the current work.  \cref{sec:validation}, where various problems
from the literature are simulated to validate the current solver.  Mixing
behaviour of spherical particles in a stirring tank is examined in
\Cref{sec:results}.



\FloatBarrier%
\section{Fluid modeling}
\label{sec:fluid-modeling}
In the current work, we follow a weakly-compressible SPH approach to model the
fluid. The continuum in SPH is modeled using particles, which have physical
properties such as mass, velocity, and these particles interact based on the
governing equations using a Guassian-like kernel
\cite{monaghan-review:2005,morris1997modeling}.


\FloatBarrier%
\subsection{Fluid governing equations}
\label{sec:fluid--governing-equations}
The fluid motion is governed by the conservation of mass and the conservation
of linear momentum, which can be expressed by the following equations:
\begin{equation}
  \label{eq:ce}
  \frac{d \rho}{d t} = - \rho \; \frac{\partial u_i}{\partial x_i},
\end{equation}
\begin{equation}
  \label{eq:me}
  \frac{d u_i}{d t} = \frac{1}{\rho} \; \frac{\partial \sigma_{ij}}{\partial x_j}
  + g_i,
\end{equation}
where $\rho$ is the density, $u_i$ is the $i$\textsuperscript{th} component of
the velocity field, $x_j$ is the $j$\textsuperscript{th} component of the
position vector, $g_i$ is the component of body force per unit mass and
$\sigma_{ij}$ is stress tensor.

The stress tensor is split into pressure and viscous parts,
\begin{equation}
  \label{eq:fluid-stress-decomposition}
  \sigma_{ij} = - p \delta_{ij} + 2 \eta \frac{\partial u_i}{\partial x_j}
\end{equation}
where $\eta$ is the dynamic viscosity of the fluid, $p$ is the pressure,
$\delta_{ij}$ is the Kronecker delta function.


\FloatBarrier%
\subsection{Discretized fluid governing equations}
\label{sec:sph--governing-equations}
The governing equations involve function, derivative and divergence
operators. In SPH, these operators are approximated based on the positions,
mass and the kernel values of the discretized particles. Assume that the
domain is discretized into N particles with mass $m_a$ and volume
$V_a$, We have
\begin{equation}
  \label{eq:mass_repr}
  m_a = V_a \> \rho_a.
\end{equation}
Based on this discretization, the discrete function approximation is given as,
\begin{equation}
  \label{eq:discrete_form}
  A_a = \sum_{b \in \text{Neigh}(a)}\> \frac{m_b}{\rho_b} A_b\> W(\ten{x}_a - \ten{x}_b, h),
\end{equation}
where $A(\boldsymbol{x}_a) = A_a$ is the value of the field property of
particle $a$, similarly for $A_b$. $\text{Neigh}(a)$ is the neighbours of
particle $a$ and $W(\ten{x}_a - \ten{x}_b, h)$ is the kernel value, with $h$
being the smoothing length. In the current work we used Quintic
spline~\cite{Violeau16} in all our simulation cases.  A symmetric derivative
approximation~\cite{Violeau16} is given as,
\begin{equation}
  \nabla A(\ten{x}_a) = \rho_a \sum_{b \in \text{Neigh}(a)} m_b \left(\frac{A_a}{\rho_a^2} + \frac{A_b}{\rho_b^2}\right) \nabla W_{ab}.
\end{equation}
A symmetric gradient operator ensures equal and opposite forces acting on the
particles.

% \subsection{Discretized fluid governing equations}
The SPH discretization of the continuity
equation~\cref{eq:ce} is given as,
\begin{equation}
  \label{eq:sph-discretization-continuity}
  \frac{{d}\rho_a}{dt} = \sum_{b} \; \frac{m_b}{\rho_{b}} \;
  \rho_{a} \; {\ten{u}}_{ab} \; \cdot \nabla_{a} W_{ab},
\end{equation}
where $\ten{u}_{ab} = \ten{u}_a - \ten{u}_b$.

%
Similarly, the discretized momentum equation is written as,
\begin{multline}
  \label{eq:sph-momentum-fluid}
  \frac{{d}\ten{u}_{a}}{dt} = - \sum_{b} m_b
  \bigg(\frac{p_a}{\rho_a^2} + \frac{p_b}{\rho_b^2} + \Pi_{ab}\bigg) \ten{I} \cdot
  \nabla_{a} W_{ab}
 \;+\;
  \sum_{b} m_b \frac{4 \eta \nabla W_{ab}\cdot
    \ten{r}_{ab}}{(\rho_a + \rho_b) (r_{ab}^2 + 0.01 h_{ab}^2)} \ten{u}_{ab}  \;+\;
  \ten{g}_{a},
\end{multline}
where $\ten{I}$ is the identity matrix. The viscous term in \Cref{eq:me} is discretized according to the
formulation introduced in
\cite{morris1997modeling}. $\Pi_{ab}$~\cite{monaghan-review:2005} is an
artificial viscosity item to maintain the stability of the numerical scheme,
given as,
\begin{align}
  \label{eq:mom-av}
  \Pi_{ab} =
  \begin{cases}
\frac{-\alpha h_{ab} \bar{c}_{ab} \phi_{ab}}{\bar{\rho}_{ab}}
  & \ten{u}_{ab}\cdot \ten{r}_{ab} < 0, \\
  0 & \ten{u}_{ab}\cdot \ten{r}_{ab} \ge 0,
\end{cases}
\end{align}
where,
%
\begin{equation}
  \label{eq:av-phiij}
  \phi_{ab} = \frac{\ten{u}_{ab} \cdot \ten{r}_{ab}}{r^2_{ab} + 0.01 h^2_{ab}},
\end{equation}
%
where $\ten{r}_{ab} = \ten{r}_a - \ten{r}_b$,
$\ten{u}_{ab} = \ten{u}_a - \ten{u}_b$, $h_{ab} = (h_a + h_b)/2$,
$\bar{\rho}_{ab} = (\rho_a + \rho_b)/2$, $\bar{c}_{ab} = (c_a + c_b) / 2$, and
$\alpha$ is the artificial viscosity parameter, is taken to be $0.1$ in the current
work. The pressure $p_a$ is evaluated using an equation of state:
\begin{equation}
\label{eqn:sph-eos}
  p_a = \rho_0 \, c_0^2 \bigg(\frac{\rho_a}{\rho_{0}} - 1 \bigg).
\end{equation}
where, $c_0=10 \times V_{\text{max}}$ is speed of sound, while $\rho_0$ as the
initial density of the particles.


\FloatBarrier%
\subsection{Boundary Conditions}
\label{sec:boundary_conditions}

The ghost particle approach proposed by \citet{Adami2012} is used to model the
boundaries. To ensure kernel completion of the fluid particle near the
boundary, we use three layers of ghost particles to model the solid wall.  The
properties of the solid wall are interpolated from the fluid particles.

When the viscous force is computed, the no slip boundary condition is used,
where the velocity ($\ten{u}_{\text{Ga}}$) on the boundary set as,
\begin{equation}
  \label{eq:no-slip-bc-u}
  \ten{u}_{\text{Ga}} = 2 \ten{u}_{\text{p}} - \ten{\hat{u}}_{\text{a}}.
\end{equation}
The projected velocity $\ten{\hat{u}}_{\text{a}}$ on the ghost particles is
computed using,
\begin{equation}
  \label{eq:v-ghost}
  \ten{\hat{u}}_a = \frac{\sum_b\ten{u}_b W_{ab}}{\sum_b W_{ab}},
\end{equation}
where $\ten{u}_b$, is the velocity of the fluid particle $b$ and $W_{ab}$ is the kernel
value between the fluid particle and the ghost particle.


The pressure of the boundary particle is extrapolated from its surrounding
fluid particles by the following equation,
\begin{equation}
  \label{eq:pressure-bc}
  p_w = \frac{\Sigma_f p_f W_{wf} + (\ten{g} - \ten{a}_{\ten{w}}) \cdot \Sigma_f
    \rho_f \ten{r}_{wf} W_{wf}}{\Sigma_f W_{wf}},
\end{equation}
where $\ten{a}_w$ is the acceleration of the wall. The subscript $f$ denotes
the fluid particles and $w$ denotes the wall particles.



\FloatBarrier%
\section{Rigid body dynamics}
\label{sec:rbd}
% The rigid body is discretized into particles with equal spacing each particle
% with mass $m_i$ and density $\rho_i$. Rigid body has a total 6 degrees of
% freedom (DOF), divided into $3$ translational and $3$ rotational.

% An approach using quaternions is given in \cite{dietemann2020smoothed}
% An another paper using quaternions \cite{guan2024numerical}


The equations governing the dynamics of a rigid body are, balance of linear and
angular momentum given by,
\begin{equation}
  \label{eq:rfc:balance_linear_mom}
  \frac{d \; (M \ten{v}_{cm})}{d t} = \sum_i \ten{F}_i + \ten{F}_{\text{contact}},
\end{equation}
\begin{equation}
  \label{eq:rfc:balance_angular_mom}
  \frac{d \ten{L}}{d t} = \teng{\tau}_{cm},
\end{equation}
where $M$, $\ten{v}_{cm}$ are the mass and velocity of the center of mass of
the rigid body.  $\ten{F}_i, \teng{\tau}_{cm}, \ten{L} $ are the force acting at
point $i$, torque and angular momentum about the center of mass of the rigid
body. Where the rigid body is being sampled using ghost SPH particles. In the
current case, force acting on the particle $i$, $\ten{F}_i$, is due to the
interaction with the fluid particles and $\ten{F}_{\text{contact}}$ is the
force acting on the solid body due to interaction with other solid bodies.
The torque $\teng{\tau}_{cm}$ and angular momentum $\ten{L}$ are computed as,
\begin{equation}
  \label{eq:rfc:torque}
  \teng{\tau}_{cm} = \sum_i \ten{F}_i \times (\ten{x}_{cm} - \ten{x}_{i}) + \ten{F}_{\text{contact}} \times (\ten{x}_{cm} - \ten{x}_{\text{contact}}),
\end{equation}
\begin{equation}
  \label{eq:rfc:moi}
  \teng{L} =
  \sum_i \; \ten{r}_i \times \; (\teng{\omega} \times \ten{r}_i)
  = \sum_i \; m_i \; [(\ten{r}_i \cdot \ten{r}_i) \ten{I} - \ten{r}_i \otimes \ten{r}_i],
\end{equation}
where $\ten{x}_{cm}$ and $\omega$ are the position of the center of mass and
angular velocity of the rigid body. $m_i$, $\ten{x}_{i}$, $\ten{r}_i$,
$\ten{x}_{\text{contact}}$ are the mass, position of ghost particle, position
of ghost particle $i$ with respect to vector center of mass, and point of
contact of two solid bodies interacting.

\begin{figure}[!htpb]
  \centering
  \includegraphics[width=0.7\textwidth]{images/rigid_body/rigid_body}
  \caption{Body frame and local frame description of rigid body}
  \label{fig:gloabl_body_frame_rb}
\end{figure}
We use two coordinate frames to capture the dynamics of the rigid body, a
global frame and a local frame as shown in
\cref{fig:gloabl_body_frame_rb}. The local fixed frame, which moves with
rigid body is always located at the center of mass ($\ten{x}_{cm}$). The
state of the rigid body at a given time ($t$) can be described using position
($\ten{x}_{cm}$) and velocity ($\ten{v}_{cm}$) of the center of mass, a
rotation matrix($\ten{R}$) to represent the orientation of the rigid body with
respect to the global frame, and angular velocity($\teng{\omega}$). The center
of mass is computed as
\begin{equation}
  \label{eq:rfc:center_of_mass}
  \ten{x}_{cm} = \frac{\sum_i m_i \; \ten{x}_{i} }{\sum_i m_i }.
\end{equation}
The position of the ghost particle ($i$) in
\cref{fig:gloabl_body_frame_rb} belonging to the rigid body at time $t$ can be
computed as,
\begin{equation}
  \label{eq:rfc:rb_particle_pos_update}
  \ten{x}_i = \ten{x}_{cm} + \ten{r}_{i},
\end{equation}
with
\begin{equation}
  \label{eq:rfc:rb_particle_pos_update}
  \ten{r}_i = \ten{R} \overline{\ten{r}}_{i},
\end{equation}
where $\overline{\ten{r}}_{i}$ is the position of the ghost particle $i$ about
the body frame axis and remains constant through out the simulation. The
rotation matrix $\ten{R}$ is used to bring the body frame position vector to
the global frame $\ten{O}$. Similarly the velocity vector of the ghost
particle is computed as,
\begin{equation}
  \label{eq:rfc:rb_particle_vel_update}
  \ten{v}_i = \ten{v}_{cm} + \teng{\omega} \times \ten{r}_{i}.
\end{equation}


The total force ($\ten{F}$) acting on a solid body can be divided into the
force acting on the ghost particle $i$ due to the interaction with the fluid,
and the force acting on the solid body due to the interaction with the other
solid body, given as
\begin{eqnarray}
  \label{eq:rfc:rb_particle_pos_update}
  \ten{F} = \sum_i \ten{F}_{\text{rfc}}^i + \ten{F}_{\text{contact}}
\end{eqnarray}
\cref{sec:dem} explains the approach to compute force
$\ten{F}_{\text{contact}}$ acting on the solid particle due to the interaction
with the rigid bodies. The force $\ten{F}_{\text{rfc}}^i$ acting due to the
interaction with the fluid particles follows \cref{sec:rfc}.


\FloatBarrier%
\subsection{Contact models}
\label{sec:dem}

\begin{figure}[!htpb]
  \centering
  \includegraphics[width=0.7\textwidth]{images/spherical_particles_dem_representation}
  \caption{Illustration of contact modelling between two rigid spherical
    particles immersed in a fluid tank.}
  \label{fig:spherical-particles-in-tank-dem}
\end{figure}

We resolve the contact among the spherical particles using the discrete
element method \cite{luding_dem_2008}. In the current work we utilized a
non-linear contact force model. In DEM, the force acting on a particle $a$ due
to the interaction with the particle $b$ is resolved into a normal and
tangential component. The normal force component represents a repulsive force,
while the tangential component is used to model the friction between the
interacting particles.  The normal force ($\teng{F}_a^{n}$) on particle $a$
due to the interaction with the particles $b$ is given by a non-linear,
Hetzian model \cite{brilliantov1996model}, including damping, is given as,
\begin{equation}
  \label{eq:contact-algorithm-normal}
  \ten{F}_a^n = k_n \delta_{n} \ten{n} - \eta_n (\ten{v}_{ab} \cdot \ten{n}) \ten{n}.
\end{equation}
Here, $\ten{v}_{ab} = \ten{v}_{a} - \ten{v}_b$ is the relative velocity of
particle $a$ with respect to the contacting particle $b$,
$\ten{n}$ normal unit vector passing from particle $b$ to $a$, and the overlap $\delta_{n}$ is computed
using the radius of particle $a$ ($R_a$) and particle $b$ ($R_b$) as
\begin{equation}
  \label{eq:cf-overlap}
  \delta_{n} = R_{a} + R_{b} - r_{ab},
\end{equation}
$k_n$ and $\eta_n$ are the normal spring stiffness and damping coefficient, which
are computed using the material properties of the bodies in contact and the
coefficient of restitution among the interacting bodies, given by:
\begin{equation}
  \label{eq:kf-stiffness}
  k_n = \frac{4}{3} \; E^{*} \; \sqrt{R^{*} \delta_n},
\end{equation}
\begin{equation}
  \label{eq:kf-stiffness}
  \eta_n = -2 \sqrt{\frac{5}{6}} \beta \; \sqrt{S_n m^*},
\end{equation}
where
\begin{equation}
  \label{eq:kf-stiffness}
  \frac{1}{E^{*}} = \frac{1 -\nu_a^2}{E_a} + \frac{1 -\nu_b^2}{E_b}
\end{equation}
\begin{equation}
  \label{eq:kf-stiffness}
  \beta = \frac{\ln{e}}{\ln{e}^2 + \pi},
\end{equation}
here, $e$ is the coefficient of restitution.
\begin{equation}
  \label{eq:kf-stiffness}
  R^{*} = \frac{R_a R_b}{R_a + R_b}
\end{equation}
\begin{equation}
  \label{eq:kf-stiffness}
  m^{*} = \frac{m_a m_b}{m_a + m_b}
\end{equation}
\begin{equation}
  \label{eq:kf-stiffness}
  S_n = 2 E^{*} \sqrt{R^{*} \delta_n}
\end{equation}







\subsection{Tangential force computation}
\label{sec:tangential-force-computation}
To handle the frictional contact, we associate a tangential spring attached to
particle $a$ and particle $b$ to compute the tangential force, which initially has
a magnitude of zero ($|\Delta \textit{\textbf{l}}_a|=0$). The tangential spring
is activated when the particle comes into contact with particle $b$. The
tangential force is history-dependent. The contact friction force is
proportional to the tangential displacement, which is integrated over
the contact time as
\begin{equation}
  \label{eq:tangential-force}
  \ten{F}_{a}^{t^{n+1}} =
  -k_t \Delta \textit{\textbf{l}}_a^{\,n + 1} - \eta_t \ten{v}_t =
  -k_t \big[\big(\Delta {\textit{\textbf{l}}}_a^{\,n} \
  + \ten{v}_{ab}^{n + 1} \Delta t\big) \cdot \ten{t}_a^{n + 1} \big] \
  \ten{t}_a^{n + 1} - \eta_t \ten{v}_t,
\end{equation}
where $\Delta t$ is the time step,
and $k_t$ is the tangential spring stiffness coefficient. The tangential unit
vector is computed by,
\begin{equation}
  \label{eq:tangential-vect}
  \ten{t}_a = \frac{\ten{v}_{ab} - (\ten{v}_{ab} \cdot \ten{n}) \ten{n}}{|\ten{v}_{ab} - (\ten{v}_{ab} \cdot \ten{n}) \ten{n}|}.
\end{equation}
The tangential spring stiffness ($k_t$) and the tangential damping coefficient
$\eta_t$ are given as:
\begin{equation}
  \label{eq:kf-stiffness}
  k_t = 8 \; G^{*} \; \sqrt{R^{*} \delta_n},
\end{equation}
\begin{equation}
  \label{eq:kf-stiffness}
  \eta_t = -2 \sqrt{\frac{5}{6}} \beta \; \sqrt{S_t m^*},
\end{equation}
with $G^*$ and $S_t$ are given as,
\begin{equation}
  \label{eq:kf-stiffness}
  \frac{1}{G^{*}} = \frac{2 (2 - \nu_a) (1 + \nu_a)}{E_a} +  \frac{2 (2 - \nu_b) (1 + \nu_b)}{E_b},
\end{equation}
\begin{equation}
  \label{eq:kf-stiffness}
  S_t = 8 G^{*} \sqrt{R^{*} \delta_n},
\end{equation}
where, $G$ is the shear modulus of the particles, $\nu$ is the Poisson's ratio.



The tangential force is coupled to the normal force through the Coulomb's law,
\begin{equation}
  \label{eq:Coulomb-law}
  \ten{F}_{a}^{t} = \min(\mu |\ten{F}_{a}^{n}|, |\ten{F}_{a}^{t}|) \
  \frac{\ten{F}_{a}^{t}}{|\ten{F}_{a}^{t}|}.
\end{equation}
This allows us to impose the sliding friction condition between the
interacting solids. Finally, the total force acting on the particle $a$ due to
the interaction with particle $b$ is:
\begin{equation}
  \label{eq:contact-force}
  \ten{F}_{a}^{\text{contact}} = \ten{F}_{a}^{n} + \ten{F}_{a}^{t}
\end{equation}

An equal and opposite force of the same magnitude is applied to
particle $b$, given as
\begin{equation}
  \label{eq:contact-force}
  \ten{F}_{b}^{\text{contact}} = - \ten{F}_{a}^{\text{contact}}.
\end{equation}



\FloatBarrier%
\section{Solid fluid coupling}
\label{sec:rfc}

% Coupling equation with single particles he2017coupled
% Coupling with dummy particles \cite{guan2024numerical}
% A different coupling equation meng2022hydroelastic

\begin{figure}[!htpb]
  \centering
  \begin{subfigure}{0.22\textwidth}
    \centering
    \includegraphics[width=1.0\textwidth]{images/rfc_explantion_schematic/real_spherical_particles}
    \subcaption{A rigid spherical particle}%\label{fig:rings:ipst-nu-0.47-0}
  \end{subfigure}\hspace{15mm}%
  \begin{subfigure}{0.24\textwidth}
    \centering
    \includegraphics[width=1.0\textwidth]{images/rfc_explantion_schematic/sph_sampled_spherical_particles}
    \subcaption{A rigid spherical particle sampled with dummy SPH particles}%\label{fig:rings:ipst-nu-0.47-1}
  \end{subfigure}
  \caption{A rigid particle being discretized into a number of dummy SPH
    particles, to handle the interaction with surrounding fluid.}
\label{fig:real_particle_sph_sampling}
\end{figure}
To calculate the force exerted on the rigid particle due to the surrounding
fluid, we employ a method involving the sampling of the spherical particle
using dummy SPH particles, depicted in
\cref{fig:real_particle_sph_sampling}. These SPH particles are evenly
distributed and remain fixed, moving in tandem with the velocity of the
solid particle at any given location.  They serve as SPH boundary particles
and contribute to the computation of fluid particle density and acceleration
according to \Cref{eq:sph-discretization-continuity,eq:sph-momentum-fluid}.
To establish the pressure of these SPH particles, we utilize the fixed ghost
particle boundary technique outlined in \cref{sec:boundary_conditions}.


With solid particles being discretized into SPH particles and immersed in
fluid can be seen in \cref{fig:many_rb_in_fluid_sph_particles}.
\begin{figure}[!htpb]
  \centering
  \includegraphics[width=0.7\textwidth]{images/rfc_zoomed_combined}
  \caption{Solid spherical particles sampled with dummy SPH particles being
    immersed in a fluid tank.}
  \label{fig:many_rb_in_fluid_sph_particles}
\end{figure}
The force on the fluid particle due to the
interaction with the sampled dummy SPH particles is considered in the momentum
\cref{eq:sph-momentum-fluid} and the continuity
\cref{eq:sph-discretization-continuity}. The force acting on the sampled dummy
SPH particle due to the interaction with the fluid is given by,
\begin{equation}
  \label{eq:rfc-force}
  \ten{F}_{\text{rfc}}^i = -m_i \sum_{f} m_f \bigg(\frac{p_f}{\rho_{f}^2} +
  \frac{p_i}{\rho_{i}^2}\bigg) \nabla_{i} W(x_{if}) +
  m_i \sum_{f} m_f \frac{4 \eta \nabla_i W_{if}\cdot
    \ten{r}_{if}}{(\rho_i + \rho_f) (r_{if}^2 + 0.01 h_{if}^2)} \ten{u}_{if}
\end{equation}
where, $m_i$ signifies the hydrodynamic mass of the sampled dummy SPH particle,
and $\rho_i$ represents its hydrodynamic density. $m_f$, $p_f$ and
$\rho_f$ are mass, pressure and density of the fluid particle.


\FloatBarrier%
\section{Time Integration}
\label{sec:integration}

The modeling of solid-solid interaction requires a smaller time step than the
fluid. We choose the minimum of these two timesteps to move the system forward
in time. For the numerical stability of fluid, the time step depends on the
Courant–Friedrichs–Lewy (CFL) condition \cite{monaghan-review:2005} as,
\begin{equation}
  \label{eq:rfc:time-step-cfl}
  \Delta t_{\text{fl}} = \mathrm{min} \bigg( 0.25 \; \frac{h}{c + |U|} ,  0.25 \; \frac{h^2}{\nu},  0.25 \; \frac{h^2}{g} \bigg),
\end{equation}
where $|U|$ is the maximum velocity magnitude, $c$ is the speed of sound
typically chosen as $10 |U|$ for fluids in this work. For solid body, the time
step is constrained \cite{cundall_discrete_1979} as,
\begin{equation}
  \label{eq:rfc:time-step-body-force}
  \Delta t_{\text{rb}} \leq \frac{\pi}{50} \sqrt{\frac{m}{k_r}}.
\end{equation}
A minimum timestep is chosen as
\begin{equation}
  \label{eq:rfc:time-step-body-force}
  \Delta t = min(\Delta t_{\text{fl}}, \Delta t_{\text{rb}}).
\end{equation}


\subsection*{Fluid particles update}
We use the kick-drift-kick scheme \cite{monaghan-review:2005} for the time
integration. We move the velocities of the fluid and the solid particles to
half time step,
\begin{equation}
  \label{eq:velocity-update-stage-1}
  \ten{u}_a^{t+\frac{1}{2} \Delta t} = \ten{u}_a^{t} + \frac{\Delta t}{2} \bigg(\frac{d\ten{u}_{a}}{dt}\bigg)^t,
\end{equation}
here, $()_a^t$ represents the properties of particle $a$ at time $t$ and
$()_a^{t+\frac{1}{2} \Delta t}$ corresponds to the halfway point between time
$t$ and $t + \Delta t$. Then the time derivative of density is calculated using
\cref{eq:sph-discretization-continuity}, with velocities at half time step
employed for the calculation.  The updated density and particle position
are determined by,
\begin{equation}
  \label{eq:density-update-stage-2}
  \rho_{a}^{t+\Delta t} = \rho_{a}^{t} + \Delta t \; \bigg(\frac{d\rho_{a}}{dt}\bigg)^{t+\frac{1}{2} \Delta t},
\end{equation}
\begin{equation}
  \label{eq:position-update-stage-2}
  \ten{r}_{a}^{t+\Delta t} = \ten{r}_{a}^{t} + \Delta t \; \ten{u}_{a}^{t+\frac{1}{2}\Delta t}.
\end{equation}
%
Finally, at new time-step particle position, the momentum velocity is updated
\begin{equation}
  \label{eq:velocity-update-stage-3}
  \ten{u}_a^{t+\Delta t} = \ten{u}_a^{t+\frac{1}{2}\Delta t} + \frac{\Delta t}{2} \bigg(\frac{d\ten{u}_{a}}{dt}\bigg)^{t+\frac{1}{2}\Delta t}.
\end{equation}


\subsection*{Solid particles update}
We evolve the state of the rigid body through the integration of the
\cref{eq:rfc:balance_linear_mom,eq:rfc:balance_angular_mom}. Following the
kick-drift-kick integration scheme, the linear velocity of the center of mass
($\ten{v}_{cm}$) and angular momentum ($\ten{L}$) are first updated to half
timestep as,
\begin{equation}
  \label{eq:rfc:lin_vel_cm_update}
  \ten{v}_{cm}^{t+\frac{1}{2} \Delta t} = \ten{v}_{cm}^{t} + \frac{\Delta t}{2} \frac{\ten{F}_{cm}}{M} \;
\end{equation}
\begin{equation}
  \label{eq:rfc:ang_mom_update}
  \ten{L}^{t+\frac{1}{2} \Delta t} = \ten{L}^{t} + \frac{\Delta t}{2} \teng{\tau}_{cm} \;.
\end{equation}
Here, $\ten{F}_{cm} = \sum_i \ten{F}_i + \ten{F}_{\text{contact}}$.

With the velocities at half time step, we update position of the center of
mass and the rotation matrix ($\ten{R}$) to the next time step as,
\begin{equation}
  \label{eq:rfc:lin_pos_cm_update}
  \ten{x}_{cm}^{t+\Delta t} = \ten{x}_{cm}^{t} + \ten{v}_{cm}^{t+\frac{1}{2} \Delta t} \; \Delta t,\\
  \ten{R}^{t+ \Delta t} = \ten{R}^{t} + \tilde{\teng{\omega}}^{t+\frac{1}{2} \Delta t} \, \ten{R}^{t} \; \Delta t,
\end{equation}
where $\tilde{\teng{\omega}}^{t+\frac{1}{2} \Delta t}$ is matrix formulation of angular
velocity $\omega$. Here, the angular velocity is computed with
\begin{equation}
  \label{eq:rfc:ang_velocity_update}
  \teng{\omega}^{t+\frac{1}{2} \Delta t} = (\textit{\teng{I}}^{-1})^{t} \; \ten{L}^{t+\frac{1}{2} \Delta t}.
\end{equation}
Here, moment of inertia at the new time step is computed as,
\begin{equation}
  \label{eq:rfc:moi_update}
  (\textit{\teng{I}}^{-1})^{t} = \ten{R}^{t} \textit{\teng{\overline{I}}}^{-1} (\ten{R}^{t})^T.
\end{equation}
where moment of inertia ($\textit{\teng{\overline{I}}}^{-1}$) in body frame is
used to compute in global frame at every time instant for faster computations.
The moment of inertia ($\textit{\teng{\overline{I}}}$) is computed as,
\begin{equation*}
\textit{\teng{\overline{I}}} =
\begin{bmatrix}
\sum_i m_i (y_i^2 + z_i^2) & -\sum_i m_i x_iy_i & -\sum_i m_i x_iz_i\\
-\sum_i m_i x_iy_i & \sum_i m_i (x_i^2 + z_i^2) &  -\sum_i m_i y_iz_i\\
-\sum_i m_i  x_iz_i & -\sum_i m_i y_iz_i & \sum_i m_i (x_i^2 + y_i^2)
\end{bmatrix}.
\end{equation*}

The position and velocity of the ghost particles comprising the rigid body are
updated based on the properties of the center of mass by
\begin{eqnarray}
  \label{eq:rfc:rb_particle_pos_update}
  \ten{r}_i = \ten{R} \cdot \overline{\ten{r}}_{i},\\
  \ten{x}_i = \ten{x}_{cm} + \ten{r}_{i},\\
  \ten{v}_i = \ten{v}_{cm} + \teng{\omega} \times \ten{r}_{i}.
\end{eqnarray}

Finally, at new time-step particle position, the velocity and angular moment
to the new time step are updated
\begin{equation}
  \label{eq:rfc:lin_vel_cm_update}
  \ten{v}_{cm}^{t+\Delta t} = \ten{v}_{cm}^{t+\frac{1}{2} \Delta t} + \frac{\Delta t}{2} \frac{\ten{F}_{cm}}{M} \;
\end{equation}
\begin{equation}
  \label{eq:rfc:ang_mom_update}
  \ten{L}^{t+\Delta t} = \ten{L}^{t+\frac{1}{2} \Delta t} + \frac{\Delta t}{2} \teng{\tau}_{cm} \;.
\end{equation}


\FloatBarrier%
\section{Model validation}
\label{sec:validation}
Initially, we validate our fluid solver through the resolution of the
Poiseuille flow problem. Subsequently, we validate the Discrete Element Method
(DEM) solver by simulating a normal head-on collision between two spherical
particles and addressing a particle-wall impact scenario.  We validate the
coupled SPH-DEM solver with simulations involved like a circular particle
entering a steady tank, a cube settling in a water tank.

\FloatBarrier%
\subsection{Poiseuille's flow}
\label{sec:poiseuille_flow}
% https://onlinelibrary.wiley.com/doi/epdf/10.1002/nag.898

\begin{table}[!ht]
  \centering
  \begin{tabular}[!ht]{ll}
    \toprule
    Quantity & Values\\
    \midrule
    $\rho$, Fluid density & $1000$ kg\,m\textsuperscript{-3} \\
    $\alpha$, Artificial viscosity & $0$ \\
    $\zeta$, Kinematic viscosity & $0.01$ m\textsuperscript{2}s\textsuperscript{-1}\\
    $dx$, Particle spacing & $\frac{1}{60}$m \\
    Time of simulation & 50 s \\
    % Gravity $[g_x, g_y, g_z]$ & $[0.0, 0.0, 0.0]$\\
    \bottomrule
  \end{tabular}
  \caption{Material and numerical parameters used for modeling of Poiseuille's
    flow with SPH.}%
  \label{tab:dem_validation_1}
\end{table}
\begin{figure}[!htpb]
  \centering
  \includegraphics[width=0.7\textwidth]{images/fluid_01_benchmark_poisuelle/poiseuille_schematic}
  \caption{Poiseuille flow problem: Schematic of a fluid being driven in
    between two parlell plates due to a pressure gradient force.}
  \label{fig:poiseuille_schematic}
\end{figure}
We study unsteady flow between two infinite, parallel plates at rest in
presence of pressure gradient to validate our fluid solver implementation.
The plates are placed $1$ m apart vertically, where the fluid is driven due to
a pressure gradient, a form of body force. The flow is towards positive x
direction. The schematic is shown in \cref{fig:poiseuille_schematic}.  In this
exact conditions, the Navier-Stokes equations admit the time dependent
solution, as given by \citet{morris1997modeling} as,
\begin{equation}
  \label{eq:poiseuille_exact_soln}
  v_x(y, t) = \frac{F}{2 \zeta}y(y - L) + \sum_{n=0}^{\infty}\frac{4FL^2}{\zeta \pi^3 (2n + 1)^3} \sin\bigg(\frac{\pi y}{L} (2 n + 1) \bigg) \exp\bigg(\frac{ (2 n + 1)^2 \pi^2 \zeta}{L^2} t\bigg)
\end{equation}
$\zeta$ is kinematic viscosity. This test case serves to
%  kinematic viscosity ($\frac{\mu}{\rho_0}$)
validate the no-slip boundary condition of our developed scheme. We use
numerical parameters such as a viscosity of $0.01$, a particle spacing of
$\frac{1}{60}$, and set the speed of sound to ten times the maximum velocity
that the fluid can attain. The simulation runs for a total duration of $50$
seconds.


\Cref{fig:poiseuille_soln_graph} depicts the variation of the u-velocity in
y-direction, compared to the analytical solution given in
\cref{eq:poiseuille_exact_soln}.  By observing
\cref{fig:poiseuille_soln_graph}, we notice a close resemblance between the
SPH simulation results and the analytical solution in
\cref{eq:poiseuille_exact_soln}, thus validating our solver.
\begin{figure}[!htpb]
  \centering
  \includegraphics[width=0.7\textwidth]{figures/plane_poiseuille_flow_2D/case_1/comparison.pdf}
  \caption{Velocity profile of the Poiseuille flow compared against the
    analytical solution at time $t=50$ seconds.}
  \label{fig:poiseuille_soln_graph}
\end{figure}



% \FloatBarrier%
% \subsection{Rigid body validation: Dzhanibekov effect on a T-shaped rigid
% body}
% \label{sec:rb_1}
% % https://doi.org/10.1063/5.0190167



\FloatBarrier%
\subsection{DEM validation 1: Normal impact of spherical particle}
\label{sec:DEM_validation_1_normal_impact}

\begin{table}[!ht]
  \centering
  \begin{tabular}[!ht]{ll}
    \toprule
    Quantity & Values\\
    \midrule
    $E$, Young's modulus & $48$ GPa \\
    $\nu$, Poisson's ratio & $0.2$ \\
    $\rho$, Density & $2800$ kg\,m\textsuperscript{-3} \\
    $\mu$, Friction coefficient & $0$ \\
    $e$, Coefficient of restitution & $1$ \\
    Time of simulation & 80 $\mu$s \\
    Gravity $[g_x, g_y, g_z]$ & $[0.0, 0.0, 0.0]$\\
    \bottomrule
  \end{tabular}
  \caption{Material parameters used for the normal impact of spherical particle.}%
  \label{tab:dem_validation_1}
\end{table}
\begin{figure}[!htpb]
  \centering
  \includegraphics[width=0.6\textwidth]{images/results_dem_1_validation_particle_particle_impact/dem_01_head_on_schematic}
  \caption{Schematic of two spherical particles of equal radius in a
    head-on collision with equal magnitude of velocity but opposite direction.}
  \label{fig:result:dem_1_schematic}
\end{figure}
To validate the Discrete Element Method (DEM) solver, we analyze the normal
collision between two spherical particles.  In
\cref{fig:result:dem_1_schematic}, the initial setup depicts both particles
moving towards each other with an initial velocity of $10$
m\,s\textsuperscript{-1}.  Each particle has a radius of $0.01$ m and is
composed of glass. The material properties of the glass material are given in
\Cref{tab:dem_validation_1}. \Cref{fig:result:dem_1_force_vs_overlap} depicts
the variation of the normal force and the amount of overlap, compared to the
analytical findings presented in \cite{chung2011benchmark}. The force curve
generated by our model closely matches the analytical result, hereby
confirming the validation of our DEM solver's model for normal contact force.
\begin{figure}[!htpb]
  \centering
  \includegraphics[width=0.8\textwidth]{figures/benchmark_1_ss_colliding_elastic/fn_overlap}
  \caption{Variation of the normal impact force with overlap of the impacting
    particles, compared to the analytical result.}
  \label{fig:result:dem_1_force_vs_overlap}
\end{figure}



\FloatBarrier%
\subsection{DEM validation 2: Particle wall oblique impact}
\label{sec:DEM_validation_2_particle_wall_impact}

\begin{table}[!ht]
  \centering
  \begin{tabular}[!ht]{ll}
    \toprule
    Quantity & Values\\
    \midrule
    $E$, Particle's young's modulus & $380$ GPa \\
    $\nu$, Particle's poisson's ratio & $0.23$ \\
    $\rho$, Particle's density & $4000$ kg\,m\textsuperscript{-3} \\
    $E$, Wall's young's modulus & $70$ GPa \\
    $\nu$, Wall's poisson's ratio & $0.25$ \\
    $\mu$, Friction coefficient between particle and the wall & $0.092$ \\
    $e$, Coefficient of restitution & $1$ \\
    Impact angle of particle & Varies \\
    Time of simulation & 60 $\mu$s \\
    gravity $[g_x, g_y, g_z]$ & $[0.0, 0.0, 0.0]$\\
    \bottomrule
  \end{tabular}
  \caption{Material parameters of the sphere and the wall in particle wall
    oblique impact problem.}%
  \label{tab:dem_validation_2}
\end{table}
\begin{figure}[!htpb]
  \centering
  \includegraphics[width=0.6\textwidth]{images/results_dem_2_validation_particle_wall_impact/schematic}
  \caption{Schematic of a spherical particle impacting a wall
    with same magnitude of velocity but different impact angles.}
  \label{fig:result:dem_sw_contact_schematic}
\end{figure}
As part of our validation test cases, we consider the collision of a spherical
particle with wall with different incident angles and a constant velocity in
magnitude. This test case is useful in testing the tangential interaction
modeling of our DEM solver.  The current problem was experimentally
investigated by \citet{kharaz2001experimental}, and is commonly employed by
various DEM solvers to validate their codes, as seen in the work of
\cite{di2004comparison} and \cite{golshan2023lethe}.  The schematic of the
body and wall is shown in \cref{fig:result:dem_sw_contact_schematic}.  The
dimensions and material properties of both the particle and the wall are set
based on the study of \cite{di2004comparison,kharaz2001experimental}. The
radius of the impacting particle is $2.5 \times 10^{-3}$ m and is made of aluminium
oxide. The material properties of both the impacting particle and the wall are
listed in \Cref{tab:dem_validation_2}.  The impact velocity has a magnitude of
$3.9$ m\,s\textsuperscript{-1}.


In \cref{fig:result:dem_sw_contact_omega_vs_theta}, we show variation of the
rebound angular velocity with the incident impact angle, compared to
experimental data from \citet{kharaz2001experimental}.  The angular velocity
variation generated by our model closely aligns with the
experimental data, thereby validating the tangential
contact model.
\begin{figure}[!htpb]
  \centering
  \includegraphics[width=0.6\textwidth]{figures/benchmark_4_sw_colliding_oblique/angle_vs_ang_vel.pdf}
  \caption{Variation of the rebound angular velocity of the impacting particle
    with the incidence angle, compared to the experimental result and Lethe DEM \cite{golshan2023lethe}.}
  \label{fig:result:dem_sw_contact_omega_vs_theta}
\end{figure}


\FloatBarrier%
\subsection{RFC validation 1: Single particle entering into a tank}
\label{sec:rfc_validation_1_single_particle_entry}
\begin{figure}[!htpb]
  \centering
  \includegraphics[width=0.4\textwidth]{images/rfc_01_skillen_2013_particle_entry_in_hs_tank/Skillen_2013_particle_entry_in_hs_tank}
  \caption{A circular cylinder entering into a calm water tank under the
    influence of the gravity.}
  \label{fig:results_rfc_01_skillen_2013}
\end{figure}
To validate the rigid-fluid coupling solver, we examine the motion of a
circular disc descending into a steady hydrostatic tank, considering two
different densities for the disc. This scenario has been investigated
experimentally by \citet{greenhow1983nonlinear} and numerically by various
studies, including \citet{sun2006water} using the Boundary Element Method and
\cite{sun2018accurate} employing the SPH technique.
\Cref{fig:results_rfc_01_skillen_2013} illustrates the initial setup, with the
cylinder positioned $0.5$ meters from the free surface. The cylinder's radius
is $0.11$ meters. Material and numertical peoperties of the fluid and the
solid particle are specified in \Cref{tab:rfc_validation_1}. The cylinder
descends due to gravity into the tank. We perform tests using three different
spacings to assess convergence: resolving the cylinder diameter into 20, 50,
and 80 particles respectively. The speed of sound is set to ten times the
maximum fluid velocity attainable.
\begin{table}[!ht]
  \centering
  \begin{tabular}[!ht]{ll}
    \toprule
    Quantity & Values\\
    \midrule
    $\rho_{\text{fluid}}$, Fluid density & $1,000$ kg\,m\textsuperscript{-3} \\
    $\rho_{\text{solid}}$, Solid density & $500$ and $1,000$ kg\,m\textsuperscript{-3} \\
    $\alpha$, Artificial viscosity & $0.0$ \\
    $\zeta$, Kinematic viscosity & $10^{-6}$ m\textsuperscript{2}s\textsuperscript{-1}\\
    $dx$, Particle spacing & $5.5 \times 10^{-3}$ m, $2.2 \times 10^{-3}$ m, and
$1.375 \times 10^{-3}$m \\
    Time of simulation & $0.16$ s \\
    Gravity $[g_x, g_y, g_z]$ & $[0.0, -9.81, 0.0]$\\
    \bottomrule
  \end{tabular}
  \caption{Material and numerical parameters used for the solid particle
    entry in a steady tank.}%
  \label{tab:rfc_validation_1}
\end{table}


\Cref{fig:result_rfc_01_result_displacement}, depicts the evolution of the
depth of penetration of the cylinder, compared against the experimental result
by \citet{greenhow1983nonlinear}, numerical techniques of boundary element
method (BEM)(\cite{sun2006water}) and delta-plus SPH
(\cite{sun2018accurate}). From \cref{fig:result_rfc_01_result_displacement},
we find that the current numerical results matches well with the experimental result as
well as the numerical studies and also converges while increasing the
resolution.
\begin{figure}[!htpb]
  \centering
  \begin{subfigure}{0.48\textwidth}
    \centering
    \includegraphics[width=1.0\textwidth]{figures/skillen_2013_water_entry_half_buoyant/penetration_vs_t}
    \subcaption{Half buoyant cylinder}%\label{fig:rings:ipst-nu-0.47-0}
  \end{subfigure}
  \begin{subfigure}{0.48\textwidth}
    \centering
    \includegraphics[width=1.0\textwidth]{figures/skillen_2013_water_entry_neutrally_buoyant/penetration_vs_t}
    \subcaption{Neutrally buoyant cylinder}%\label{fig:rings:ipst-nu-0.47-1}
  \end{subfigure}
  \caption{Variation of the penetration depth of the cylinder with time,
    compared to BEM model \cite{sun2006water} and $\delta$+SPH model\cite{sun2018accurate}.}
\label{fig:result_rfc_01_result_displacement}
\end{figure}


\FloatBarrier%
\subsection{RFC validation 2: A 3D cube falling in water}
\label{sec:rfc_validation_2_falling_solid_in_water}

The current test case involves falling of a solid cube in a calm water
tank. The position of the cube is compared with the
experimental\cite{wu2014two} result for the validation of the current SPH-DEM
solver. This problem has been used as a standard benchmark in validating
several other numberical techniques, such as in VOF-DEM solver
\cite{nan2022cfd} and other SPH-DEM solvers \cite{qiu20173d}. The cube has a
side length of $20$ mm. While the dimentions of the water tank is 150 mm
$\times$ 140 mm $\times$ 140 mm. While the water depth is taken as $131$ mm. The cube
center is initially placed at a height of $131$ mm, immersed half way in
water. The density of the water is taken as $1,000$ kg\,m\textsuperscript{-3},
and the density of the cube is $2,120$ kg\,m\textsuperscript{-3}. The schemtic
of the initial configuration is shown in \cref{fig:results_rfc_02_falling}. We
show that the current solver can handle 3D problems with this test case.
\begin{figure}[!htpb]
  \centering
  \includegraphics[width=0.4\textwidth]{images/rfc_02_falling_solid_in_water/schematic}
  \caption{Schematic of a cube of density $2,120 kg\,m\textsuperscript{-3}$
    immersed half way in a steady hydrostatic tank is allowed to settle under
    gravity.}
  \label{fig:results_rfc_02_falling}
\end{figure}



\Cref{fig:results_rfc_02_y_vs_t} shows the evolution of the y-position of
center of mass of the cube, compared against the experimental result by
\citet{wu2014two}. From \cref{fig:results_rfc_02_y_vs_t}, the comparison
reveals a close correspondence between the results obtained by our solver and
the experimental findings, affirming its capability to handle fluid-solid
coupling problems efficiently.
\begin{figure}[!htpb]
  \centering
  \includegraphics[width=0.6\textwidth]{figures/wu_2014_falling_solid_3d/z_vs_t}
  \caption{Vertical position variation of the center of mass of the cube with
    time, compared against the experimental result by \citet{wu2014two}.}
  \label{fig:results_rfc_02_y_vs_t}
\end{figure}

% \FloatBarrier%
% \subsection{RFC validation 3: Falling solid in water}
% \label{sec:rfc_validation_3_falling_solid_in_water}

% zhang2009simulation
% \cite{wu2014two} experimental.
% \cite{qiu20173d} SPH-DEM
% \cite{nan2022cfd} using VOF-DEM


% \FloatBarrier%
% \subsection{RFC validation 3: Force acting on particle in a shear flow}
% \label{sec:rfc_validation_3_single_particle_in_shear_flow}


% \FloatBarrier%
% \subsection{RFC testing 1: Falling circular body in water}
% \label{sec:rfc_testing_1_falling_circular_body_in_water}

% We take a circular body and change the viscosity of the fluid and check the





\FloatBarrier%
\section{Results and discussion}
\label{sec:results}
We investigate the mixing behavior of spherical particles under a stirrer,
examining cases with varying particle diameters and stirrer speeds.

\subsection{Mixing of spherical particles in a fluid tank}
\label{sec:mixing-spherical-particles-in-fluid-tank-homogeneous}
In this section, we investigate particle dispersion within a tank under
various stirrer velocities. Each circular particle has a radius of $0.055$ m,
with a total of $54$ particles. The tank are set to a length of $3.3$ m and a
height of $1.65$ m. The stirrer, positioned on the left side, has a lenght of
$0.75$ m, half of which is submerged into the fluid, and a width of $0.11$
m. After allowing the particles to settle for $1$ second, the stirrer
oscillates throughout the fluid length. We examine three stirrer speeds: $1$,
$3$ m\,s\textsuperscript{-1}.  The schematic of the figure is given in
\Cref{fig:schematic-Dinesh-mixer-homogeneous}. The solid particles have a
Young's modulus of $10^{9}$ N\,m\textsuperscript{-2}, a Poisson's ratio of
$0.23$, a density of $1050$ kg\,m\textsuperscript{-3}, and a friction
coefficient of $0.5$ is assumed among the solid particles.
\begin{figure}[!htpb]
  \centering
  \includegraphics[width=0.45\textwidth]{figures/dinesh_2024_mixing_with_stirrer_homogeneous_2d/case_1/time0}
  \caption{ Initial configuration of particles in a fluid tank including the
    stirrer.}
  \label{fig:schematic-Dinesh-mixer-homogeneous}
\end{figure}


\Cref{fig:1-mixing-1,fig:1-mixing-2,fig:1-mixing-3}, illustrate the
distribution of particles within a fluid tanker under the influence of a
stirrer.  In \Cref{fig:1-mixing-1}, particles tend to aggregate and settle
gradually into a clump at the center. Additionally, their motion aligns with
the surrounding fluid motion. Conversely, \Cref{fig:1-mixing-2,fig:1-mixing-3}
depict a scenario where the high speed of the stirrer causes fluid to
circulate heavily at the tank's corners, resulting in particle
settlement at the tank's corners without efficient mixing.
\begin{figure}[!htpb]
  \centering
  \begin{subfigure}{0.48\textwidth}
    \centering
    \includegraphics[width=1.0\textwidth]{figures/dinesh_2024_mixing_with_stirrer_homogeneous_2d/case_1/time1}
    \subcaption{t = $3$ s}\label{fig:1-mixing-1-a}
  \end{subfigure}
  \begin{subfigure}{0.48\textwidth}
    \centering
    \includegraphics[width=1.0\textwidth]{figures/dinesh_2024_mixing_with_stirrer_homogeneous_2d/case_1/time2}
    \subcaption{t = $6$ s}\label{fig:1-mixing-1-b}
  \end{subfigure}

  \begin{subfigure}{0.48\textwidth}
    \centering
    \includegraphics[width=1.0\textwidth]{figures/dinesh_2024_mixing_with_stirrer_homogeneous_2d/case_1/time3}
    \subcaption{t = $9.9$ s}\label{fig:1-mixing-1-c}
  \end{subfigure}
  % \begin{subfigure}{0.45\textwidth}
  %   \centering
  %   \includegraphics[width=1.0\textwidth]{figures/dinesh_2024_mixing_with_stirrer_homogeneous_2d/case_1/time0}
  %   \subcaption{t = $7.38$ ms}\label{fig:1-mixing-1-d}
  % \end{subfigure}
  \caption{Snapshots of fluid, stirrer and the rigid circular particles at
    three time steps, where the stirrer is oscillating at a velocity of $1$
    m\,s\textsuperscript{-1}. The colour contour of the fluid particles
    represents the velocity magnitude.}
\label{fig:1-mixing-1}
\end{figure}
\begin{figure}[!htpb]
  \centering
  \begin{subfigure}{0.48\textwidth}
    \centering
    \includegraphics[width=1.0\textwidth]{figures/dinesh_2024_mixing_with_stirrer_homogeneous_2d/case_2/time1}
    \subcaption{t = $3$ s}\label{fig:1-mixing-1-a}
  \end{subfigure}
  \begin{subfigure}{0.48\textwidth}
    \centering
    \includegraphics[width=1.0\textwidth]{figures/dinesh_2024_mixing_with_stirrer_homogeneous_2d/case_2/time2}
    \subcaption{t = $6$ s}\label{fig:1-mixing-1-b}
  \end{subfigure}

  \begin{subfigure}{0.48\textwidth}
    \centering
    \includegraphics[width=1.0\textwidth]{figures/dinesh_2024_mixing_with_stirrer_homogeneous_2d/case_2/time3}
    \subcaption{t = $9.9$ s}\label{fig:1-mixing-1-c}
  \end{subfigure}
  \caption{Snapshots of fluid, stirrer and the rigid circular particles at
    three time steps, where the stirrer is oscillating at a velocity of $3$
    m\,s\textsuperscript{-1}. The colour contour of the fluid particles
    represents the velocity magnitude.}
\label{fig:1-mixing-2}
\end{figure}
\begin{figure}[!htpb]
  \centering
  \begin{subfigure}{0.48\textwidth}
    \centering
    \includegraphics[width=1.0\textwidth]{figures/dinesh_2024_mixing_with_stirrer_homogeneous_2d/case_3/time1}
    \subcaption{t = $3$ s}\label{fig:1-mixing-1-a}
  \end{subfigure}
  \begin{subfigure}{0.48\textwidth}
    \centering
    \includegraphics[width=1.0\textwidth]{figures/dinesh_2024_mixing_with_stirrer_homogeneous_2d/case_3/time2}
    \subcaption{t = $6$ s}\label{fig:1-mixing-1-b}
  \end{subfigure}

  \begin{subfigure}{0.48\textwidth}
    \centering
    \includegraphics[width=1.0\textwidth]{figures/dinesh_2024_mixing_with_stirrer_homogeneous_2d/case_3/time3}
    \subcaption{t = $9.9$ s}\label{fig:1-mixing-1-c}
  \end{subfigure}
  % \begin{subfigure}{0.45\textwidth}
  %   \centering
  %   \includegraphics[width=1.0\textwidth]{figures/dinesh_2024_mixing_with_stirrer_homogeneous_2d/case_1/time0}
  %   \subcaption{t = $7.38$ ms}\label{fig:1-mixing-1-d}
  % \end{subfigure}
  \caption{Snapshots of fluid, stirrer and the rigid circular particles at
    three time steps, where the stirrer is oscillating at a velocity of $5$
    m\,s\textsuperscript{-1}. The colour contour of the fluid particles
    represents the velocity magnitude.}
\label{fig:1-mixing-3}
\end{figure}


\FloatBarrier%
\subsection{Mixing of spherical particles of variable size in a fluid tank}
\label{sec:mixing-spherical-particles-in-fluid-tank-inhomogeneous}
In this section, we analyze mixing of circular
particles with two different radii. These particles have a ratio of $1.2$
between the larger and smaller particle, with the smaller particle having a
radius of $0.055$ m. The schematic of the figure is provided in
\Cref{fig:schematic-Dinesh-mixer-inhomogeneous}
\begin{figure}[!htpb]
  \centering
  \includegraphics[width=0.45\textwidth]{figures/dinesh_2024_mixing_with_stirrer_inhomogeneous_2d/case_1/time0}
  \caption{Schematic of mixer with two different radius in a fluid
    tank.}\label{fig:schematic-Dinesh-mixer-inhomogeneous}
\end{figure}

\Cref{fig:2-mixing-1,fig:2-mixing-2} shows the distribution of particles in a
fluid tank under the influence of a stirrer, with two stirrer speeds at three
different time steps. Similar to
\Cref{sec:mixing-spherical-particles-in-fluid-tank-homogeneous}, from
\Cref{fig:2-mixing-1}, we see that with a lower stirrer speed, particles tend
to aggregate and settle gradually into a clump at the center and at higher
speeds due to the generated circulation near the tank corners, the particles
settle at the corner and do not involve in the mixing, as can be seen from
\Cref{fig:2-mixing-2}.
\begin{figure}[!htpb]
  \centering
  \begin{subfigure}{0.48\textwidth}
    \centering
    \includegraphics[width=1.0\textwidth]{figures/dinesh_2024_mixing_with_stirrer_inhomogeneous_2d/case_1/time1}
    \subcaption{t = $3$ s}\label{fig:1-mixing-1-b}
  \end{subfigure}
  \begin{subfigure}{0.48\textwidth}
    \centering
    \includegraphics[width=1.0\textwidth]{figures/dinesh_2024_mixing_with_stirrer_inhomogeneous_2d/case_1/time2}
    \subcaption{t = $6$ s}\label{fig:1-mixing-1-c}
  \end{subfigure}

  \begin{subfigure}{0.48\textwidth}
    \centering
    \includegraphics[width=1.0\textwidth]{figures/dinesh_2024_mixing_with_stirrer_inhomogeneous_2d/case_1/time3}
    \subcaption{t = $9.9$ s}\label{fig:1-mixing-1-d}
  \end{subfigure}
  \caption{Snapshots of fluid, stirrer and the rigid circular particles at
    three time steps, where the stirrer is oscillating at a velocity of $1$
    m\,s\textsuperscript{-1}. The colour contour of the fluid particles
    represents the velocity magnitude.}
\label{fig:2-mixing-1}
\end{figure}
\begin{figure}[!htpb]
  \centering
  \begin{subfigure}{0.48\textwidth}
    \centering
    \includegraphics[width=1.0\textwidth]{figures/dinesh_2024_mixing_with_stirrer_inhomogeneous_2d/case_2/time1}
    \subcaption{t = $3$ s}\label{fig:1-mixing-1-b}
  \end{subfigure}
  \begin{subfigure}{0.48\textwidth}
    \centering
    \includegraphics[width=1.0\textwidth]{figures/dinesh_2024_mixing_with_stirrer_inhomogeneous_2d/case_2/time2}
    \subcaption{t = $6$ s}\label{fig:1-mixing-1-c}
  \end{subfigure}

  \begin{subfigure}{0.48\textwidth}
    \centering
    \includegraphics[width=1.0\textwidth]{figures/dinesh_2024_mixing_with_stirrer_inhomogeneous_2d/case_2/time3}
    \subcaption{t = $9.9$ s}\label{fig:1-mixing-1-d}
  \end{subfigure}
  \caption{Snapshots of fluid, stirrer and the rigid circular particles at
    three time steps, where the stirrer is oscillating at a velocity of $3$
    m\,s\textsuperscript{-1}. The colour contour of the fluid particles
    represents the velocity magnitude.}
\label{fig:2-mixing-2}
\end{figure}

% \FloatBarrier%
% \subsection{Mixing of spherical particles in a fluid tank with turbine impeller}
% \label{sec:mixing-spherical-particles-in-tank}

% Study of free-surface and solids suspension in top-sealed tanks stirred by
% pitched blade turbine impellers through DEM-VOF method
% This paper has an example of mixing problem.


\FloatBarrier%
\section{Conclusions}
\label{sec:conclusions}
In this study, the particle mixing efficiency in a two-dimensional fluid tank
due to a stirrer is examined, employing a fully resolved SPH-DEM solver. The
fluid phase is represented using a weakly compressible SPH formulation, while
the interaction and dynamics of spherical particles are modeled using the
Discrete Element Method (DEM). The interaction between the fluid and discrete
particles is modelled by associating the spherical particles with surrogate
SPH particles. The coupled SPH-DEM solver is developed within the PySPH
software framework \cite{ramachandran2021pysph} and throughly validated.  The
influence of stirrer speed on particle dispersion in a 2D tank was thoroughly
investigated. Different stirrer speeds were examined, revealing distinct
behaviours. At lower stirrer speeds, particles tended to aggregate and remain
in center within the tank, influenced due to the surrounding fluid
velocity. While, higher stirrer speeds generated fluid circulation, causing
some particles to settle at the tank corners, thus hindering their mixing. The
same behaviour was found even in scenarios involving particles with different
radii.


Due to the current solver's capability in handling the free surfaces, we can
extend the current to study the particle dispersion in a 3D cylindrical tank
featuring an impeller. Further, we can explore how the speed and placement of
the impeller affect mixing efficiency. Additionally, influence of particle
cohesion on the inter-particle mixing behavior can be examined.

% \section*{References}


\bibliographystyle{model6-num-names}
\bibliography{references}
\end{document}

% ============================
% Table template for reference
% ============================
% \begin{table}[!ht]
%   \centering
%   \begin{tabular}[!ht]{ll}
%     \toprule
%     Quantity & Values\\
%     \midrule
%     $L$, length of the domain & 1 m \\
%     Time of simulation & 2.5 s \\
%     $c_s$ & 10 m/s \\
%     $\rho_0$, reference density & 1 kg/m\textsuperscript{3} \\
%     Reynolds number & 200 \& 1000 \\
%     Resolution, $L/\Delta x_{\max} : L/\Delta x_{\min}$ & $[100:200]$ \& $[150:300]$\\
%     Smoothing length factor, $h/\Delta x$ & 1.0\\
%     \bottomrule
%   \end{tabular}
%   \caption{Parameters used for the Taylor-Green vortex problem.}%
%   \label{tab:tgv-params}
% \end{table}

%%% Local Variables:
%%% mode: latex
%%% TeX-master: "paper"
%%% fill-column: 78
%%% End:
